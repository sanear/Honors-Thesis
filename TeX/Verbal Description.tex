\documentclass[notitlepage,reqno]{amsart}
% Packages I'm using
\usepackage{amsmath,amsfonts,amsthm,amssymb}
\usepackage{fullpage}
\usepackage{enumerate}
\usepackage{framed}
\usepackage{setspace}
\usepackage{fancyhdr}
\usepackage{graphicx}
\usepackage{fancyvrb}
\usepackage{color}
\usepackage{listings}

% listings package parameters
\lstset{
  numbers=left,
  basicstyle=\ttfamily\smaller,
  breaklines=true
}

% Homework-specific information
\newcommand{\hmwkTitle}{Eight Evolutionary Games on the Lattice}
\newcommand{\hmwkAuthor}{Andrea Sannier}
%\newcommand{\hmwkSubTitle}{}n
%\newcommand{\hmwkDueDate}{}
\newcommand{\hmwkClass}{}
\newcommand{\hmwkClassTime}{}
\newcommand{\hmwkInstructor}{Dr. Nicolas Lanchier}


% Commonly-used stuff
\newcommand{\pr}{\mathbb{P}}
\newcommand{\R}{\mathbb{R}}
\newcommand{\N}{\mathbb{N}}
\newcommand{\seqx}{\{x_n\}_{n=0}^\infty}
\newcommand{\seqy}{\{y_n\}_{n=0}^\infty}
\newcommand{\seqa}{\{a_n\}_{n=0}^\infty}
\newcommand{\seqb}{\{b_n\}_{n=0}^\infty}

% Allowing easy statement of theorems and lemmas (requires amsthm)
\newtheoremstyle{normal}
	{3pt}			%space above
	{3pt}			%space below
	{}				%body font
	{\parindent}	%indent amount
	{\itshape}		%theorem head font
	{}				%punctuation after theorem head
	{ }		%space after theorem head
	{}				%theorem head specification(??)
\newtheorem{thm}{Theorem}
\newtheorem*{thm*}{Theorem}
\newtheorem{lem}{Lemma}
\newtheorem*{lem*}{Lemma}

% Margins and other formatting
%\topmargin=0in
%\leftmargin=-1in
%\rightmargin=-1in
\setlength{\parindent}{0in}
\fvset{xleftmargin=2em}		%fancyvrb package allows indentation of all Verbatim environments

% Setup the header and footer
\pagestyle{fancy}
\pagenumbering{empty}

\lhead{\hmwkAuthor}
\chead{\hmwkClass}
\rhead{\hmwkInstructor\ / \hmwkTitle}
\cfoot{}
\addtolength{\headheight}{12pt}
\setlength{\headsep}{0.125in}
\renewcommand\headrulewidth{0.4pt}
\renewcommand\footrulewidth{0pt}

%%%%%%%%%%%%%%%%%%%%%%%%%%%%%%%%%%%%%%%%%
% Make Title
\title{\huge{\textmd{\textbf{\hmwkClass\\\ \hmwkTitle}}}}
\date{}
\author{\textbf{\hmwkAuthor} \\
\textmd{\hmwkInstructor}}
%%%%%%%%%%%%%%%%%%%%%%%%%%%%%%%%%%%%%%%%%

\begin{document}

\thispagestyle{fancy}

\subsection*{Abstract}
Traditionally, Evolutionary Game Theory models are based on mean-field
model ordinary differential equations and, therefore, on two
unrealistic assumptions: (1) that players are well-mixing, i.e., that
any two players can interact with equal probability, and (2) that each
interaction is deterministic. This thesis intends to shed light on a
way to correct these two assumptions by reimagining the models,
accounting for space and stochasticity. We place countably many players on an
$\mathbb{Z}^d$ lattice and allow them to play games only with some set
of nearest neighbors on each axis. Moreover, the games they play are
not deterministic and are instead governed by one of eight stochastic
processes that depend on the fitness of the neighboring players. In
this way, we discover a number of surprising results that differ from
those found in traditional models, perhaps most notably that, in some
Prisoner's Dilemma situations, cooperation is the most successful strategy.

\subsection*{Attempt at verbal description}
We first establish the notion of the infinite lattice. We have a grid
of countably many sites in $d$ dimensions, with the same cardinality
of sites along each axis, indexed by the natural
numbers. We let $\eta_t:\mathbb{Z}^d\to \{1,2\}$ be a function describing the state
of our process at time $t$. That is, $\eta_t (x)$ is the strategy of
the player at site $x$ and time $t$, which we indicate as either
strategy 1 or strategy 2. We also define $N_i:\mathbb{Z}^d \times
\mathbb{Z}\to \mathbb{Z}$ such that $N_i(x,t)$ is the number of players
employing strategy $i$ at site $x$ and time $t$. Thus, $N_i(x,t) =
|\{y\sim x:\eta_t(y)=i\}|$, where $y\sim x$ indicates that site $y$ is
a neighbor of site $x$. With these two functions in hand, we update
each site at rate ???


\end{document}
